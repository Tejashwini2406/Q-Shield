\documentclass{article}
\usepackage[utf8]{inputenc}
\usepackage{amsmath}
\usepackage{graphicx}
\usepackage{hyperref}
\title{Q-Shield Project Proposal}
\author{Tejashwini2406}
\date{2026-02-11}
\begin{document}
\maketitle

\section{Introduction}
The Q-Shield project aims to develop a comprehensive solution for quantum shielding technologies, addressing potential vulnerabilities in quantum computing systems. This proposal outlines the objectives, methodologies, and expected outcomes of the project.

\section{Objectives}
\begin{itemize}
    \item Identify quantum vulnerabilities.
    \item Develop shielding techniques.
    \item Test the effectiveness of shielding.
\end{itemize}

\section{Methodology}
The project will adopt a systematic approach:
\begin{enumerate}
    \item Research existing quantum vulnerabilities.
    \item Design shielding mechanisms.
    \item Implement and test solutions in controlled environments.
\end{enumerate}

\section{Expected Outcomes}
The expected outcomes of the Q-Shield project include:
\begin{itemize}
    \item A detailed report on quantum vulnerabilities.
    \item Development of innovative shielding technologies.
    \item Publication of findings in relevant journals.
\end{itemize}

\section{Conclusion}
In conclusion, the Q-Shield project offers significant potential to enhance the security of quantum computing systems. We invite stakeholders to collaborate and support this innovative venture.
\end{document}